\section*{Abstract}

Especially in daily rush-hour-scenarios, a street-network requires enough capacity to support the amount of drivers.
On the other hand, a street-network of too much capacity would be inefficient outside of rush-hour-scenarios.
To improve the variety of routes, such that this overload during rush-hour spreads more over the network, alternative routes in multicriteria settings are computed.
Many previous approaches need too much parameter-tuning or simply lack in their computational complexity, their needed runtime or the diversity of found routes.
This thesis presents a combination of existing approaches to create a new penalizing metric, such that popular metrics are compensated.
The approach to use a new metric allows every routing-algorithm, that is capable of dealing with multicriteria-routes, to process this new metric without further changes.
This metric is used in an existing method for computing alternative multicriteria-routes, which is enumerating personalized routes, to distribute found routes successfully over the network under holding user-provided tolerances for preferred metrics.
The results using the new metric are compared between Dijkstra and the used method for enumerating personalized routes, both on street-networks from OpenStreetMap.
To speed the route-queries significantly up, the underlying graphs of the networks are contracted via an existing realization of multicriteria contraction-hierarchies, where the contraction is supported by a linear program.