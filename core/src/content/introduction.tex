\chapter{Introduction}
\label{chap:introduction}

\todo{TODO @Florian: Opinion about introduction?}

Let assume drivers driving to somewhere a little further away, like work.
A navigation-system, asked for the fastest route, usually returns a route over larger streets, like highways.
When many drivers receive the same answer from their independent navigation-systems, the traffic tends to overload these larger streets.
These larger streets probably have multiple lanes, but allows a higher speed-limit than smaller streets, resulting in a high driver-capacity and hence a high throughput.
In addition, due to rush-hours, the traffic is not balanced over the day.
If the network shouldn't be empty most of the time, it does not match perfectly for rush-hour-scenarios, hence all these aspects leads to traffic-jams and former fast routes become inefficient.
Besides environmental impact and circumstances to handle the daily amount of rush-hour-traffic, traffic-jams in large (capital) cities cost days of the people's time and hence much money per year.~\cite{inrix:traffic-cost}
At the latest, this will be interesting with many self-driving cars, where many routes will be needed.
It would be inefficient to send all vehicles over the same streets.
So distributing this amount of traffic over the network is and will be important to save both, time and money.

This thesis explains, how a given street-network of multidimensional \glspl{metric} (travel-distance, travel-time, artificial) can be augmented, such that diverse alternative paths can be found and used to spread the load (of a typical workload) while limiting the detours of individual \glspl{stpair}.
Although the presented methods can be transferred to different kinds of networks, this thesis focusses on street-networks.
A different kind of network can be wireless-networks or the Internet, where the focus lays more on nodes than edges.
Here, data-packages are transmitted much faster than drivers are driving through street-networks and traffic-jams occur in nodes, not in the edges.

\section{Related Work}

    Totally different approaches do solve these issues, having their pros and cons.

    \subsection{Dynamic routing}

        One approach to find different routes for the same \gls{stpair} is dynamic routing.
        Dynamic (or adaptive) routing describes the ability of vehicles finding new paths based on current conditions or circumstances, while driving along another path.
        So in opposite to static routing, dynamic routing processes volatile data and processes it (more or less) immediately.
        For example, Google has gathered~\cite{barth:google-traffic} anonymous data from Android-users to predict current traffic-jams in the street-network.
        Since most smartphones use Android~\cite{kantar:android-vs-ios}, there is enough data being collectable for sufficient accuracy.
        Although, this approach requires a global observer, including infrastructure, being able to process incoming queries in reasonable time.
        With the drivers' data, like their location, the global observer can compute heuristics to approximate the current workload of other network-parts.
        Usually, this cost additional knowledge-exchange between the global observer (\eg\ Google) and the drivers.
        If the drivers aren't willing or able to share relevant information, the infrastructure has to be extended to collect relevant data.
        The need of extending the infrastructure makes the routing dependent on the infrastructure and could cost money.

    \subsection{Static routing}

        Another approach to overcome the issue of providing distributed routes is solving this issue statically.
        Statically in this context means calculating the routes with help of non-volatile data.
        So, routing-computations are not dependent on a street-network's state, but on its attributes.
        Paths are calculated before a vehicle drives it, and the vehicles' path doesn't change over time.
        That's why a static approach is less dependent on dynamic traffic and more dependent on the street-network itself, in opposite to the dynamic approaches.

        There are several static computation-methods to compute a selection of several routes with respect to one \gls{metric}, that are sufficiently distinct.
        One idea is the computation of the $k$ shortest paths for a given \gls{stpair}.
        According to~\cite{eppstein:finding_k_shortest_paths}, the computation is complex while resulting paths don't differ much.
        Another quite intuitive idea is the penalization of popular routes to reduce their usage before computing the shortest path again.
        The focus in \cite{bader:alternative-route-graphs} lays on measuring the quality of alternative paths via node-degrees and using heuristics for computing multidimensional shortest paths.
        In \cite{chondrogiannis:k_shortest_paths}, the problem is addressed, that paths of a set of found $k$ shortest paths for a given \gls{stpair} should differ from eachother, not only from the shortest path itself.
        Hence they build the set iteratively and discard unwished paths.
        To this end, a similarity between paths is defined.
        However, both approaches suffer from the sensitivity of tuning the parameters and graphs are considered to have one \gls{metric}.

        When it comes to graphs of multidimensional \glspl{metric}, pareto-optimal solutions occur implicitely.
        A first idea is computing all pareto-optimal paths.
        As stated in~\cite{delling:pareto-paths}, computing all pareto-optimal paths is too expensive to be practical at all.
        That's why~\cite{barth:alternative_multicriteria_routes} develops an algorithm enumerating \glspl{personalized_route}, extended by a geometric interpretation of shortest paths in their underlying \gls{cost}-space.
        To speed the queries up, a preprocessing-technique called \gls{contraction-hierarchies} is applied in their paper.
        In addition, while the original approach of contraction-hierarchies is developed in~\cite{geisberger:contraction_hierarchies}, an improved version~\cite{funke:personal-routes} using linear programming is used.
        Further, to reduce the search-space, the enumeration is restricted by a clever check in~\cite{barth:alternative_routes} without influencing its correctness or completeness.

\section{Main contribution}

    The contribution of this paper is the combination of penalization and \gls{personalized_routing}.
    A preprocessing-phase is added, that analyzes the street-network with the help of the (restricted) enumeration from~\cite{barth:alternative_multicriteria_routes}.
    This preprocessing leads to a new \gls{metric}, penalizing edges of high quality/popularity.
    Applying \gls{personalized_routing} with all interesting \glspl{metric}, including the new one(s), leads to a more distributed set of routes.
    Thus, the street-network is used in a broader manner.
    At the same time, due to \gls{personalized_routing}, routing-queries can be processed as quickly as before.
    To guarantee a certain tolerance towards the shortest path, a filter is added to the enumeration from~\cite{barth:alternative_multicriteria_routes}.

\section{Outline}

    \todo{TODO overview of chapters}