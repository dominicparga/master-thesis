\chapter{Introduction}
\label{chap:introduction}

\todo{TODO Template okay like this? E.g. line-spacing etc}

Let assume drivers driving to somewhere a little further away, like work.
A navigation-system, asked for the fastest route, usually returns a route over larger streets, like highways.
When many drivers receive the same answer from their independent navigation-systems, the traffic tends to overload these larger streets.
These larger streets probably have multiple lanes, but allows a higher speed-limit than smaller streets, resulting in a high driver-capacity and hence a high throughput.
In addition, due to rush-hours, the traffic is not balanced over the day.
If the network shouldn't be empty most of the time, it does not match perfectly for rush-hour-scenarios, hence all these aspects leads to traffic-jams and former fast routes become inefficient.
Besides environmental impact and circumstances to handle the daily amount of rush-hour-traffic, traffic-jams in large (capital) cities cost days of the people's time and hence much money per year.~\cite{inrix:traffic-cost}
At the latest, this will be interesting with many self-driving cars, where many routes will be needed.
It would be crucial to send all vehicles over the same streets.
So distributing this amount of traffic over the network is and will be important to save both, time and money.

\section{Related Work}

    Totally different approaches do solve these issues, having their pros and cons.

    \subsection{Dynamic routing}

        To solve this issue, dynamic routing can be used.
        For example, Google has gathered~\cite{barth:google-traffic} anonymous data from Android-users to predict current traffic-jams in the street-network.
        Since most smartphones use Android~\cite{kantar:android-vs-ios}, there is enough data being collectable for sufficient accuracy.
        Although, this approach requires a global observer, including infrastructure.
        It also needs data-packages, in street-networks represented by drivers, able to or willing to share their location in the network and being processed in reasonable time or even real-time.
        For example, in wireless-networks, data-packages are transmitted much faster than drivers in street-networks and traffic-jams occur in nodes, not in the nodes' connections.
        In such cases, heuristics can be used to approximate the current workload of other network-parts, but this cost throughput and/or capacity (hence money).
        While users would wait for a few seconds until a good route is found, users are not that tolerant when it comes to network-traffic (e.g.\ search-queries via Internet).
        However, in this thesis, only street-networks are covered.

    \subsection{Alternative routing}

        Another approach, solving this issue statically, is looking for alternative routes, that should distribute the suggested routes in a broader manner.
        In opposite to the dynamic approaches, a static approach is less dependent on dynamic traffic and more dependent on the (street-) network itself.

        There are several computation-methods described in the related-work-section of~\cite{barth:alternative_routes}.
        According to~\citeS{1}{barth:alternative_routes}, these methods lack in quality or practicability, which is depicted shortly in the following.
        On one side, you may compute multiple shortest paths for the same source and destination.
        Multiple shortest paths, computed with respect to the same criterion, are too similar.
        If they are not too similar, their computation gets more complex or more unhandy while avoiding the similarity.
        Hence, artificially \todo{TODO wording} penalizing popular routes is more practically, but also needs more parameter-tuning.
        In both cases, less similarity while getting more alternative paths leads to raising computation-time.
        \todo{TODO Is this preprocessing-time? If no, my contribution is nicer, because it has high computation-time only during preprocessing}

        To avoid these issues, \cite{barth:alternative_routes} develops an algorithm called \textit{cyclops} based on \textit{personalized routing}, extended by a geometric interpretation of shortest paths in their underlying cost-space.
        This algorithm is used in its optimized version from~\cite{barth:alternative_multicriteria_routes}, where \textit{cyclops} is combined with \textit{contraction-hierarchies}~\cite{geisberger:contraction_hierarchies}.
        \todo{TODO Is this the original paper? Because it is not the first one describing contraction, see~\cite{schultes:route-planning}.}

\section{Main Contribution}

    In general, the static penalization tends to shift the problem (of choosing the same routes) from one metric to another.
    The contribution of this paper is the combination of penalization and alternative-routing.
    A preprocessing-phase is added to the approach of~\cite{barth:alternative_multicriteria_routes} analyzing the street-network with help of the optimized \textit{cyclops}-exploration.
    This leads to a new metric, penalizing edges of high quality/popularity (measured by counting suggested routes by \textit{Dijkstra}).
    Here, the approvement over other penalization-strategies is the combination of penalization with the ability of \textit{cyclops} to find many different routes.
    Further, \textit{cyclops} can be adjusted with a tolerance for every used metric, that allows to guarantee a bad path being not worse than the best path's cost plus tolerance.

    Penalizing in a preprocessing in combination with the \textit{cyclops}-routing leads to a more distributed set of good, alternative routes, while the routing-query can be processed as quickly as before (using \textit{cyclops} or \textit{Dijkstra} with \textit{personalized routing}).

\section{Outline}

    \todo{TODO overview of chapters}