\chapter{Preliminaries}
\label{chap:preliminaries}

\todo{TODO Writing in a style like each subsection is independent of the whole thesis? Like a dictionary, if someone doesn't know some technique? -> Florian says no $:)$}
\todo{TODO add some text here}

\section{Graphs and paths}

    \todo{TODO write}

\section{Dijkstra's algorithm (bidirectional)}

    \todo{TODO Explain only CH-Query}

    \todo{TODO write}

    \subsection{Correctness-proof of bidirectional Dijkstra}

        \todo{TODO Remove proof if not needed in the thesis (probably not needed).}

        \todo{
            TODO The termination of the bidirectional Astar is based on the first node v, that is marked by both, the forward- and the backward-subroutine.
            However, this common node v is part of the shortest path s->t wrt to this particular hop-distance H, but doesn't have to be part of the shortest path s->t wrt to edge-weights.

            Every node, that is not settled in any of the both subroutines, has a longer distance to both s and t than the already found common node v and hence can not be part of the shortest path (wrt to edge-weights).
            Otherwise, it would have been settled before v since the priority-queues sort by weights.
            In other words, only already settled nodes and their neighbors (which are already enqueued) can be part of the shortest path.

            In conclusion, emptying the remaining nodes in the queues and picking the shortest path of the resulting common nodes leads to the shortest path wrt to edge-weights from s to t.
        }

        \todo{%
            TODO extend proof onto A* with \gls{contraction-hierarchies}: Here, the proof for bidirectional \gls{dijkstra} doesn't hold, because each sub-graph doesn't visit every node of the total graph, due to the level-filter when pushing edges to the queue.
            Hence, the forward- and the backward-query are not balanced wrt weights.
            Thus, after finding the first meeting-node, the hop-distance of the shortest-path could be arbitrary.
            This leads to wrong paths with normal bidirectional \gls{dijkstra}.
            To correct this issue, stop the query after polling a node of a sub-distance, which is higher than the currently best meeting-node's total distance.
        }

\section{Personalized routing}

    \todo{TODO write}
    \Gls{personalized_routing} reduces the multidimensional metrics via dot-product with a preference-vector to one dimension.


\section{Contraction-hierarchies}

    \todo{TODO write}

    \subsection{Overview}

    \todo{TODO write}

    \subsection{Computing shortcuts of multiple metrics with LP/ILP (TODO title)}

    \todo{TODO write}

\section{Cyclops}

    \todo{TODO write}

    \subsection{Idea}

        \todo{TODO write; Here could be talked about the approach of removing the nd-triangulation and how cyclops takes this idea/context and makes it correct by building convex-hull explicitly (because new facets could destroy old facets).}

    \subsection{TODO Proof of correctness + completeness}

    \todo{TODO write}