\chapter{Preliminaries}
\label{chap:preliminaries}

\todo{TODO Writing in a style like each subsection is independent of the whole thesis? Like a dictionary, if someone doesn't know some technique?}

% - Dijkstra, no A* because multiple metrics vs heuristic
% - pesonalized routing
% - Contraction-Hierarchies
% - Multi-CH-Construction
%   - LP/ILP from Florian's alternative routes
% - cyclops
%   - trying to improve, but failed
%   - other context for cyclops
%   - build convex-hull explicit because new facets could destroy old facets

\todo{TODO add some text here}

    \section{Street-network as offset-graph (TODO: wrong name)}

    \todo{TODO add some text here}

    \section{Routing-algorithms with bidirectional A* and Contraction-Hierarchies}

    \todo{TODO add some text here}

        \subsection{Bidirectional Dijkstra}

        \todo{TODO add some text here}

        \subsection{Correctness-proof of bidirectional Dijkstra}

        The termination of the bidirectional Astar is based on the first node v, that is marked by both, the forward- and the backward-subroutine.
        However, this common node v is part of the shortest path s->t wrt to this particular hop-distance H, but doesn't have to be part of the shortest path s->t wrt to edge-weights.

        Every node, that is not settled in any of the both subroutines, has a longer distance to both s and t than the already found common node v and hence can not be part of the shortest path (wrt to edge-weights).
        Otherwise, it would have been settled before v since the priority-queues sort by weights.
        In other words, only already settled nodes and their neighbors (which are already enqueued) can be part of the shortest path.

        In conclusion, emptying the remaining nodes in the queues and picking the shortest path of the resulting common nodes leads to the shortest path wrt to edge-weights from s to t.

        \todo{%
            TODO extend proof onto A* with contraction-hierarchies: Here, the proof for bidirectional Dijkstra doesn't hold, because each sub-graph doesn't visit every node of the total graph, due to the level-filter when pushing edges to the queue.
            Hence, the forward- and the backward-query are not balanced wrt weights.
            Thus, after finding the first meeting-node, the hop-distance of the shortest-path could be arbitrary.
            This leads to wrong paths with normal bidirectional Dijkstra.
            To correct this issue, stop the query after polling a node of a sub-distance, which is higher than the currently best meeting-node's total distance.
        }
        \subsection{Contraction-Hierarchies}

    \section{Finding alternative routes via modified personalized routing (TODO: Titel okay?)}

    \todo{TODO add some text here}

        \subsection{Correctness-proof of algorithm}

        \todo{TODO add some text here}

        \subsection{Completeness-proof of algorithm}

        \todo{TODO add some text here}