\chapter{Preparations}
\label{chap:preparations}

    \section{Proof of bidirectional A*}

    The termination of the bidirectional Astar is based on the first node v, that is marked by both, the forward- and the backward-subroutine.
    However, this common node v is part of the shortest path s->t wrt to this particular hop-distance H, but doesn't have to be part of the shortest path s->t wrt to edge-weights.

    Every node, that is not settled in any of the both subroutines, has a longer distance to both s and t than the already found common node v and hence can not be part of the shortest path (wrt to edge-weights).
    Otherwise, it would have been settled before v since the priority-queues sort by weights.
    In other words, only already settled nodes and their neighbors (which are already enqueued) can be part of the shortest path.

    In conclusion, emptying the remaining nodes in the queues and picking the shortest path of the resulting common nodes leads to the shortest path wrt to edge-weights from s to t.