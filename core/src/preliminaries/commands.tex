%%%%
% general
%
% pretty BibTeX
\def\BibTeX{{%
    \rm B%
    \kern-.05em{\sc i\kern-.025em b}%
    \kern-.08em%
    T\kern-.1667em%
    \lower.7ex%
    \hbox{E}%
    \kern-.125emX%
}}%
%
% todos
%\newcommand{\todo}[1]{}
%\renewcommand{\todo}[1]{{\color{red} TODO: {#1}}}
%
% EN: from hmks makros.tex - \indexify
\newcommand{\toindex}[1]{\index{#1}#1}
%
% DE: wird fuer Tabellen benötigt (z.B. >{centering\RBS}p{2.5cm} erzeugt einen zentrierten 2,5cm breiten Absatz in einer Tabelle
\newcommand{\RBS}{\let\\=\tabularnewline}
%
% DE: Seitengrößen
\newcommand{\largepage}{\enlargethispage{\baselineskip}}
\newcommand{\shortpage}{\enlargethispage{-\baselineskip}}
%%%

%%%%
% highlight text
\newcommand{\define}[1]{\textit{#1}}
\newcommand{\emphasize}[1]{\textbf{#1}}
%
%%%

%%%%
% typographic
%\newcommand{\eg}[0]{e.g.}
%\newcommand{\Eg}[0]{E.g.}
%\newcommand{\ie}[0]{i.e.}
%\newcommand{\Ie}[0]{I.e.}
%
% DE: typoraphisch richtige Abkürzungen
% (xspace is not needed in every case, e.g., in this a case)
%\newcommand{\zB}{z.\,B.\xspace}
%\newcommand{\bzw}{bzw.\xspace}
%\newcommand{\usw}{usw.\xspace}
%\renewcommand{\dh}{d.\,h.\xspace}
%
% initialism (For Example -> FE or F.E.)
\newcommand{\initialism}[1]{%
\ifdeutsch%
    \textsc{#1}\xspace%
\else%
    \textlcc{#1}\xspace%
\fi%
}
\newcommand{\OMG}{\initialism{OMG}}
\newcommand{\BPEL}{\initialism{BPEL}}
\newcommand{\BPMN}{\initialism{BPMN}}
\newcommand{\UML}{\initialism{UML}}
%%%

%%%%
% maths
%
% Replace $|x|$ with $\abs{x}$ since lines' size is scaling automatically with size of x
\newcommand{\abs}[1]{\left\lvert#1\right\rvert}
%
% EN: To avoid issues with Springer's \mathplus
%     See also http://tex.stackexchange.com/q/212644/9075
\providecommand\mathplus{+}
%
% DE: Tipp aus "The Comprehensive LaTeX Symbol List"
\newcommand{\dotcup}{\ensuremath{\,\mathaccent\cdot\cup\,}}
%%%

%%%%
% algorithms
%
% EN: For the algorithmic package
\newcommand{\commentchar}{\ensuremath{/\mkern-4mu/}}
\algrenewcommand{\algorithmiccomment}[1]{\hfill $\commentchar$ #1}
%%%

%%%%
% quotation
%
% DE
\newcommand{\citeS}[2]{\cite[S.~#1]{#2}}
\newcommand{\citeSf}[2]{\cite[S.~#1\,f.]{#2}}
\newcommand{\citeSff}[2]{\cite[S.~#1\,ff.]{#2}}
\newcommand{\vgl}{vgl.\ }
\newcommand{\Vgl}{Vgl.\ }
%
% EN: natbib compatibility
%\newcommand{\citep}[1]{\cite{#1}}
%\newcommand{\citet}[1]{\citeauthor{#1} \cite{#1}}
% EN: Beginning of sentence - analogous to cleveref - important for names such as "zur Muehlen"
%\newcommand{\Citep}[1]{\cite{#1}}
%\newcommand{\Citet}[1]{\Citeauthor{#1} \cite{#1}}
\newcommand{\citeafter}[2]{\cite[according to p.~#1]{#2}}
\newcommand{\quotes}[1]{``#1''}
%%%
